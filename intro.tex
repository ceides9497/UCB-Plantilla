\begin{intro}


Las promesas de collections se crearon para optimizar la composicion de operaciones de colections. Las operaciones se pueden optimizar bajo un conjunto de reglas de inferencia \cite{juampi}.

Este es un ejemplo de citaci\'on \cite{clay_2009}.

\section*{Antecedentes}
\addcontentsline{toc}{section}{\protect\numberline{}Antecedentes}%



%para referenciar se pone el path de la imagen
Ejemplo de como hacer referencia a una imagen \figref{imagenes/UCB-Escudo}

% 0.5 significa que reducira el tama�o de la imagen a la mitad.
\largefig{h}{0.5}{imagenes/UCB-Escudo}{Ejemplo de Figura}




%se pone el ide de la tabla... el id es definido por el tag \label
Ejemplo de como hacer referencia a una tabla \ref{categorias}

%EJEMPLO DE TABLA

\begin{table}[t]
\centering
\begin{tabular}{llrrrr}
\hline
\multicolumn{2}{|l|}{\textbf{Cambios de C\'odigo Fuente}}                                                                                          
& \multicolumn{1}{r|}{\textbf{R}} 
& \multicolumn{1}{r|}{\textbf{I}} 
& \multicolumn{1}{r|}{\textbf{R/I}} 
& \multicolumn{1}{r|}{\textbf{Total}} \\ \hline
1	& Method call additions									& 	23 	& 	0  	& 	1	&	24 (29\%)  	\\
2	& Method call swaps										& 	15	& 	9 	&	0	&	24  (29\%)\\
 \hline
 
	& {\bf Total}											&	52	&	28	&	4	&	84 (100\%)	 \\
\end{tabular}
\caption{Ejemplo de Tabla}
\label{categorias}
\end{table}


\subsection*{Ejemplo de subsecci\'on}


\section*{Problema}
\addcontentsline{toc}{section}{\protect\numberline{}Antecedentes}%

\subsection*{\emph{Situaci\'on Problematica}}
\addcontentsline{toc}{subsection}{\protect\numberline{} \emph{Situacion Problematica}}%



\subsection*{\emph{Formulaci\'on del Problema}}
\addcontentsline{toc}{subsection}{\protect\numberline{} \emph{Formulaci\'on del Problema}}%



\section*{Objectivos}
\addcontentsline{toc}{section}{\protect\numberline{}Objectivos}%



\subsection*{Objetivo General}
\addcontentsline{toc}{subsection}{\protect\numberline{}Objetivo General}%


\subsection*{Objetivo Espec\'ificos}
\addcontentsline{toc}{subsection}{\protect\numberline{}Objetivo Espec\'ificos}%


\section*{Alcances}
\addcontentsline{toc}{section}{\protect\numberline{}Alcances}%



\section*{Limites}
\addcontentsline{toc}{section}{\protect\numberline{}Limites}%




\section*{Justificaci\'on}
\addcontentsline{toc}{section}{\protect\numberline{}Justificaci\'on}%


\section*{Cronograma}
\addcontentsline{toc}{section}{\protect\numberline{}Cronograma}%

%\section*{Subsecci\'on 1}
%\addcontentsline{toc}{subsection}{\protect\numberline{}Subsecci\'on 1}%


\end{intro}
